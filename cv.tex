\documentclass[11pt,a4paper]{moderncv}

\usepackage[french]{babel}
\usepackage[T1]{fontenc}
\usepackage[top=10mm, bottom=10mm, left=10mm, right=10mm]{geometry}
\usepackage[none]{hyphenat}
\usepackage[utf8]{inputenc}

\moderncvtheme[blue]{classic}

\firstname{Pierre}
\familyname{Gillet}
\address{125 allée des bastides}{83470, S\up{t}-Maximin-la-S\up{te}-Baume} 
\phone[mobile]{06\,09\,89\,44\,22}
\email{pierre.gillet@linuxw.info}
%% \homepage{floss.win}
\social[linkedin]{pierre-gillet-884393144}
\social[github]{pierregillet}
%% \social[stackexchange]{8394677}
%% \extrainfo{19\,ans}

%% \photo[64pt]{img/photo_profil.jpg}

%% \title{Je souhaite poursuivre mes études jusqu'au doctorat.}

\begin{document}

\maketitle

\vspace{-10mm}

\section{Projet professionnel}
Passionné d'informatique et de culture libre, je souhaite exercer le
métier d'enseignant-chercheur en informatique.

\section{Compétences et connaissances}
\textbf{Spécialisées}\newline
\cvitem{Langages}{
  \textbf{Rust (apprentissage)},
  \textbf{Python},
  \textbf{Java},
  \textbf{C++},
  C,
  Bash,
  GLSL (bases),
  algorithmique
  %% Kotlin (bases),
  %% Assembleur (bases),
  %% JavaScript,
  %% PHP,
}

\cvitem{Frameworks et \\ bibliothèques}{
  \textbf{GTK+},
  \textbf{Qt},
  \textbf{SDL2},
  \textbf{SFML},
  JavaFX,
  %% Boost,
  %% Bokeh,
  %% NumPy,
  %% PyGame,
  OpenGL,
  %% GLEW,
  %% GLM,
  %% Bootstrap,
  %% Django,
  %% Flask,
  Symfony 3,
  Scikit-learn
}

\cvitem{Développement}{
  \textbf{Design patterns},
  \textbf{Git} (compétences avancées),
  débuggers,
  CMake,
  UML,
  %% modèles conceptuels de données, 
  intégration continue (bases),
  tests unitaires et fonctionnels,
  complexité,
  modélisation CSP et SAT
  %% GitHub\,/\,GitLab,
  %% Meson
  %% GNU Make\,/\,Autotools
  %% Maven
}

\cvitem{Systèmes}{
  GNU\,/\,Linux (\textbf{Arch Linux}, Debian),
  Docker,
  Vagrant,
  Salt
  %% VirtualBox
}

\cvitem{Réseaux}{
  \textbf{Nginx},
  Apache,
  DNS,
  DHCP,
  routage,
  6to4
  %% Spanning Tree Protocol
}

\cvitem{Bases de données}{
  SQL,
  % PL\,/\,SQL,
  MariaDB\,/\,MySQL,
  % SQLite,
  % Triggers,
  % Vues,
  PHPMyAdmin,
  MySQL Workbench
}

\cvitem{IDE\,/\,Éditeurs}{
  \textbf{VS Code},
  \textbf{IDE JetBrains},
  % \textbf{IntelliJ IDEA},
  % PyCharm,
  % CLion,
  % PhpStorm,
  Qt Creator,
  \textbf{Emacs},
  Vim
}

\vspace{4mm}

\textbf{Associées\newline}
\cvitem{Langues}{
  %% Anglais (CEFR niveau B1 ({\em Cambridge English Preliminary}), Bachelor en Écosse),\newline
  %% Espagnol niveau débutant
  Anglais -- score de 106\,/\,120 au \emph{TOEFL iBT} le 2019-12-07, Bachelor en Écosse
}

\cvitem{Autres}{
  \textbf{\LaTeX},
  % droit,
  méthodes agiles
  %% comptabilité,
  %% économie,
  %% gestion d'entreprises,
}

\section{Expérience professionnelle et projets personnels}
\cventry{2020}{Stage de 2 mois}{Laboratoire d'Informatique \& Systèmes (LIS) de Marseille}{}{}{
  Travail sur un système de calcul argumentatif en IA symbolique.% (équipe LIRICA).
}

\cventry{2019}{Stage de 2 mois}{Laboratoire d'Informatique \& Systèmes (LIS) de Marseille}{}{}{
  Travail sur un solveur max-SAT en logique modale des hypothèses.% (équipes CANA et LIRICA).
}

\cventry{2018\,-\,2019}{Projets durant le Bachelor}{pédagogiques}{}{}{
  \begin{itemize}
  \item Conception et réalisation d'un moteur de jeu écrit en C++ utilisant OpenGL et SDL2.
  \item Implémentation de l'algorithme de recherche de chemin IDA*.
  \item Création de plusieurs effets graphiques, notamment de «\,l'alpha blending\,» et des ombres.
  \end{itemize}
}

\cventry{2017\,-\,2018}{Chargé de projet}{Agence Boeki}{Aix-en-Provence}{DUT en alternance}{
  Réalisation d'un système de gestion de caisse en ligne avec le «\,framework\,» Symfony 3.
}

% \cventry{2016}{Projet de terminale (ISN)}{pédagogique continué en projet personnel, binôme}{}{}{
%   Réalisation d'un jeu écrit en C/C++, utilisant la bibliothèque graphique SDL.
% }

%% \cventry{2017}{OpenItinerary (projet tutoré)}{pédagogique, hexanôme}{}{}{
%%   Conception d'un service de calculs d'itinéraires basé sur les cartes OpenStreetMap.
%% }

%% \cventry{2016}{Projet C++}{pédagogique, quadrinôme}{}{}{
%%   Réalisation d'un jeu en terminal écrit en C++, utilisant la bibliothèque ncurses.
%% }

%% \cventry{2016}{Projet web}
%%         {pédagogique, pentanôme}{}{}{
%%           Réalisation d'un site web en HTML5, CSS3 et PHP, utilisant Gettext pour traduire du contenu.
%%         }

\section{Formation}
\cventry{2019--2021}{Master Informatique}{\newline Parcours Intelligence Artificielle et Apprentissage Automatique}{Aix-Marseille Université}{}{}

\cventry{2018--2019}{Bachelor en «\,Computer Games Technology\,»}{\newline University of the West of Scotland}{Paisley (Écosse)}{}{}

\cventry{2018}{DUT Informatique en alternance}{IUT AMU}{Aix-en-Provence}{}{}

\cventry{2016}{Baccalauréat Scientifique, série SI, spécialité ISN}{}{}{\mbox{mention \em{bien}}}{}

% \newpage

\section{Centres d'intérêt}
\cvitem{\textbf{Veille technologique}}{
  GNU\,/\,Linux,
  programmation,
  Libre et Open Source,
  Reddit,
  communications sécurisées,
  vie privée,
  science et industrie spatiale,
  transition énergétique,
  jeux vidéos
  %% crypto-monnaies (Bitcoin)
}

\cvitem{\textbf{Événements}}{
  Global Game Jam 2019 (Glasgow),
  Hack to the Future 2017,
  Nuit de l'Info 2017
}

% \cvitem{\textbf{Logiciel libre}}{
%   Contributions au développement de logiciels libres
% }

\cvitem{\textbf{Escalade}}{
  Pratique compétitive au niveau régional (2012--2015),
  diplôme d'Initiateur SAE délivré par la FFME (2016)
}
        
\end{document}

%%% Local Variables:
%%% mode: latex
%%% TeX-master: t
%%% End:
