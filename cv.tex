\documentclass[11pt,a4paper]{moderncv}
\usepackage[utf8]{inputenc}
\usepackage[french]{babel}
\usepackage[T1]{fontenc}
\usepackage[top=0.6cm, bottom=0.6cm, left=1.0cm, right=1.0cm]{geometry}

\makeatletter
% commands
\renewcommand*{\makecvhead}{%
  % recompute lengths (in case we are switching from letter to resume, or vice versa)
  \recomputecvlengths%
  % optional detailed information (pre-rendering)
  \@initializebox{\makecvheaddetailsbox}%
  \if@details%
  \def\phonesdetails{}%
  \collectionloop{phones}{% the key holds the phone type (=symbol command prefix), the item holds the number
    \protected@edef\phonesdetails{\phonesdetails\protect\makenewline\csname\collectionloopkey phonesymbol\endcsname\collectionloopitem}}%
  \def\socialsdetails{}%
  \collectionloop{socials}{% the key holds the social type (=symbol command prefix), the item holds the link
    \protected@edef\socialsdetails{\socialsdetails\protect\makenewline\csname\collectionloopkey socialsymbol\endcsname\collectionloopitem}}%
  \savebox{\makecvheaddetailsbox}{%
    \addressfont\color{color2}%
    \if@left\begin{tabular}[b]{@{}r@{}}\fi%
    \if@right\begin{tabular}[b]{@{}l@{}}\fi%
    \ifthenelse{\isundefined{\@addressstreet}}{}{\makenewline\addresssymbol\@addressstreet%
      \ifthenelse{\equal{\@addresscity}{}}{}{\makenewline\@addresscity}% if \addresstreet is defined, \addresscity and addresscountry will always be defined but could be empty
      \ifthenelse{\equal{\@addresscountry}{}}{}{\makenewline\@addresscountry}}%
    \phonesdetails% needs to be pre-rendered as loops and tabulars seem to conflict
    \ifthenelse{\isundefined{\@email}}{}{\makenewline\emailsymbol\emaillink{\@email}}%
    \ifthenelse{\isundefined{\@homepage}}{}{\makenewline\homepagesymbol\httplink{\@homepage}}%
    \socialsdetails% needs to be pre-rendered as loops and tabulars seem to conflict
    \ifthenelse{\isundefined{\@extrainfo}}{}{\makenewline\@extrainfo}%
    \end{tabular}
  }\fi%
  % optional photo (pre-rendering)
  \@initializebox{\makecvheadpicturebox}%
  \savebox{\makecvheadpicturebox}{%
    \ifthenelse{\isundefined{\@photo}}%
               {}%
               {%
                 \if@left%
                 \hspace*{\separatorcolumnwidth}\fi%
                 \color{color1}%
                 \setlength{\fboxrule}{\@photoframewidth}%
                 \ifdim\@photoframewidth=0pt%
                 \setlength{\fboxsep}{0pt}\fi%
                 \framebox{\includegraphics[width=\@photowidth]{\@photo}}}%
               \if@right%
               \hspace*{\separatorcolumnwidth}\fi}%
  % name and title (pre-rendering)
  \@initializelength{\makecvheaddetailswidth}\settowidth{\makecvheaddetailswidth}{\usebox{\makecvheaddetailsbox}}%
  \@initializelength{\makecvheadpicturewidth}\settowidth{\makecvheadpicturewidth}{\usebox{\makecvheadpicturebox}}%
  \ifthenelse{\lengthtest{\makecvheadnamewidth=0pt}}% check for dummy value (equivalent to \ifdim\makecvheadnamewidth=0pt)
             {\setlength{\makecvheadnamewidth}{\textwidth-\makecvheaddetailswidth-\makecvheadpicturewidth}}%
             {}%
             \@initializebox{\makecvheadnamebox}%
             \savebox{\makecvheadnamebox}{%
               \begin{minipage}[b]{\makecvheadnamewidth}%
                 \if@left\raggedright\fi%
                 \if@right\raggedleft\fi%
                 \namestyle{\@firstname\ \@lastname}%
                 \\{\small 19\,ans, permis B, véhicule personnel}
                 \ifthenelse{\equal{\@title}{}}{}{\\[1.25em]\titlestyle{\@title}}%
             \end{minipage}}%
             % rendering
             \if@left%
             \usebox{\makecvheadnamebox}%
             \hfill%
             \llap{\usebox{\makecvheaddetailsbox}}% \llap is used to suppress the width of the box, allowing overlap if the value of makecvheadnamewidth is forced
             \usebox{\makecvheadpicturebox}\fi%
             \if@right%
             \usebox{\makecvheadpicturebox}%
             \rlap{\usebox{\makecvheaddetailsbox}}% \llap is used to suppress the width of the box, allowing overlap if the value of makecvheadnamewidth is forced
             \hfill%
             \usebox{\makecvheadnamebox}\fi%
             \\[2.5em]%
             % optional quote
             \ifthenelse{\isundefined{\@quote}}%
                        {}%
                        {{\centering\begin{minipage}{\quotewidth}\centering\quotestyle{\@quote}\end{minipage}\\[2.5em]}}%
                        \par
}% to avoid weird spacing bug at the first section if no blank line is left after \makecvhead

\makeatother

%%%%%%%%%%%%%%%%%%%%%%%%%%%%%%%%%%%%%%%%%%%%%%%%%%%%%%%%%%%%

\moderncvtheme[blue]{classic}

\firstname{Pierre}
\familyname{Gillet}
%% \address{Cité Universitaire Arc-de-Meyran}{13100 Aix-en-Provence} 
\address{125 allée des bastides}{83470, S\up{t}-Maximin-la-S\up{te}-Baume} 
\phone[mobile]{06\,09\,89\,44\,22}
\email{pierre.gillet@linuxw.info}
%% \homepage{floss.win}
\social[linkedin]{pierre-gillet-884393144}
\social[github]{pierregillet}
%% \social[stackexchange]{8394677}
%% \extrainfo{19\,ans}

%% \photo[64pt]{img/photo_profil.jpg}

\title{
  \large
  Je souhaite réaliser le Bachelor «\,Video Game Development\,»  à UWS.
}

\usepackage{etoolbox}
%% Pour aligner le nom au milieu de la description
%% \patchcmd{\makecvhead}{{minipage}[b]}{{minipage}[t]}{}{}
%% \patchcmd{\makecvhead}{{tabular}[b]}{{tabular}}{}{}
%% \patchcmd{\recomputecvbodylengths}{\par\addvspace{2.5ex}}{\par\addvspace{4ex}}{}{}

\begin{document}

\maketitle

\vspace{-2mm}
\section{Projet professionnel}
Passionné d'Informatique et de culture libre,
je souhaite poursuivre les études jusqu'au doctorat, et exercer le métier d'enseignant-chercheur en Informatique.

\section{Compétences et connaissances}
\textbf{Spécialisées\newline}
\cvitem{Programmation}{
  \textbf{Python},
  \textbf{Java},
  \textbf{C++},
  PHP,
  Symfony~3,
  Bash,
  JavaScript,
  C,
  %% Assembleur (MIPS),
  %% algorithmique,
  Flask,
  SlimFramework
}

\cvitem{Développement}{
  \textbf{Git (gestion de version)},
  débuggers,
  UML,
  modèles conceptuels de données, 
  intégration continue,
  tests unitaires,
  GitHub,
  GitLab,
  Gogs,
  PHPUnit,
  Maven,
  Make
}

\cvitem{Systèmes}{
  GNU\,/\,Linux (\textbf{Arch Linux}, Debian),
  Nginx,
  Apache 2
}

\cvitem{Réseaux}{
  DNS,
  DHCP,
  routage
  %% Spanning Tree Protocol
}

\cvitem{Bases de données}{
  SQL,
  PL\,/\,SQL,
  MariaDB\,/\,MySQL,
  Triggers
}

\cvitem{IDE\,/\,Éditeurs}{
  \textbf{IntelliJ IDEA},
  \textbf{PyCharm},
  PhpStorm,
  Qt Creator,
  \textbf{Emacs},
  Vim
  %% Subliminal,
  %% Atom
}

\cvitem{Outils}{
  \textbf{JavaFX},
  \textbf{GTK+},
  PHPMyAdmin,
  MySQL Workbench,
  VirtualBox,
  Docker
}

\textbf{Associées\newline}
\cvitem{Langues}{
  Anglais (CEFR niveau B1 validé par l'examen {\em Cambridge English Preliminary}),\newline
  Espagnol niveau débutant
}

%% \cvitem{Bureautique}{
%%   LibreOffice
%% }

\cvitem{Autres}{
  \textbf{\LaTeX},
  Comptabilité,
  Droit,
  Économie,
  Gestion d'entreprises,
  Méthodes Agiles
}

\section{Expérience professionnelle\,/\,projets personnels}
\cventry{2017\,-\,2018}{Chargé de projet}
        {Agence Boeki}{Aix-en-Provence}{alternance}{
          Réalisation d'un système de gestion de caisse en ligne avec le «\,framework\,» Symfony 3.
        }

\cventry{2017}{Projet tutoré}
        {pédagogique, hexanôme}{}{}{
          Conception d'un service de calculs d'itinéraires basé sur les cartes OpenStreetMap.\newline
          Gestion de projet, et travail en groupe, et réalisation d'une IHM avec la bibliothèque JavaFX.
        }

\cventry{2016}{Projet C++}
        {pédagogique, quadrinôme}{}{}{
          Réalisation d'un jeu en terminal écrit en C++, utilisant la bibliothèque ncurses.
        }

\cventry{3 mois\\2016}{Projet d'ISN}
        {pédagogique continué en projet personnel, trinôme}{}{}{
          Réalisation d'un jeu écrit en C/C++, utilisant la bibliothèque graphique SDL.
          Nous avons poursuivi ce projet en dehors du cadre scolaire après l'avoir rendu.
        }
%% \cventry{2016}{Projet web}
%%         {pédagogique, pentanôme}{}{}{
%%           Réalisation d'un site web en HTML5, CSS3 et PHP, utilisant Gettext pour traduire du contenu.
%%         }

\section{Formation}
\cventry{2016\,--\,actuel}{Préparation d'un diplôme Universitaire de Technologie en Informatique}{IUT}{\newline Aix-en-Provence}{}{2\textsuperscript{de} année en alternance}

\cventry{2016}{Baccalauréat Scientifique, série Sciences de l'Ingénieur}{}{}{\mbox{mention \em{bien}}}{Spécialité Informatique et Sciences du Numérique}

\section{Centres d'intérêt}
\cvitem{\textbf{Événements}}{
  Hack to the Future,
  Nuit de l'Info 2017
}

\cvitem{\textbf{Veille technologique}}{
  GNU\,/\,Linux,
  programmation,
  libre\,/\,Open Source,
  Reddit,
  crypto-monnaies (Bitcoin),\newline
  communications sécurisées,
  vie privée
}

\cvitem{\textbf{Logiciel libre}}{
  Contributions au développement de logiciels libres
}

\cvitem{\textbf{Escalade}}{
  Pratique compétitive au niveau régional,
  diplôme d'Initiateur SAE délivré par la FFME
}
        
\end{document}
