\documentclass[11pt,a4paper]{moderncv}
\usepackage[utf8]{inputenc}
\usepackage[french]{babel}
\usepackage[T1]{fontenc}
\usepackage[top=0.6cm, bottom=0.6cm, left=1.0cm, right=1.0cm]{geometry}
 
\moderncvtheme[blue]{classic}
%% \nopagenumbers{}

\firstname{Pierre}
\familyname{Gillet}
%% \address{Cité Universitaire Arc-de-Meyran}{13100 Aix-en-Provence} 
\address{}{13100, Aix-en-Provence} 
\phone[mobile]{06\,09\,89\,44\,22}
\email{pierre.gillet@linuxw.info}
%% \homepage{floss.win}
\social[linkedin]{pierre-gillet-884393144}
\social[github]{pierregillet}
%% \social[stackexchange]{8394677}
\extrainfo{19\,ans}

%% \photo[64pt]{img/photo_profil.jpg}

%% \title{\parbox{11cm}{\large Étudiant en DUT Informatique en alternance, je compte poursuivre des études longues.}}

\usepackage{etoolbox}
\patchcmd{\makecvhead}{{minipage}[b]}{{minipage}[t]}{}{}
\patchcmd{\makecvhead}{{tabular}[b]}{{tabular}}{}{}

\begin{document}

\maketitle

\vspace{-14mm}
\section{Projet professionnel}
Je souhaite poursuivre les études jusqu'au doctorat, et exercer le métier d'enseignant-chercheur en Informatique.

\section{Compétences et connaissances}
\textbf{Spécialisées\newline}
\cvitem{Programmation}{
  \textbf{Python},
  \textbf{Java},
  \textbf{C++},
  PHP,
  Symfony,
  Bash,
  JavaScript,
  C,
  Assembleur (MIPS),
  algorithmique,
  Flask,
  SlimFramework
}

\cvitem{Développement}{
  \textbf{Git (gestion de version)},
  débuggers,
  UML,
  modèles conceptuels de données, 
  intégration continue,
  tests unitaires,
  GitHub,
  GitLab,
  Gogs,
  PHPUnit,
  Maven,
  Make
}

\cvitem{Systèmes}{
  GNU\,/\,Linux (\textbf{Arch Linux}, Debian),
  Nginx,
  Apache 2
}

\cvitem{Réseaux}{
  DNS,
  routage,
  Spanning Tree Protocol,
  DHCP
}

\cvitem{Bases de données}{
  SQL,
  PL\,/\,SQL,
  MariaDB\,/\,MySQL,
  Triggers
}

\cvitem{IDE\,/\,Éditeurs}{
  \textbf{IntelliJ IDEA},
  \textbf{PyCharm},
  PhpStorm,
  Qt Creator,
  \textbf{Emacs},
  Vim,
  Subliminal,
  Atom
}

\cvitem{Outils}{
  \textbf{JavaFX},
  \textbf{GTK+},
  PHPMyAdmin,
  MySQL Workbench,
  VirtualBox,
  Docker
}

\textbf{Associées\newline}
\cvitem{Langues}{
  Anglais (CEFR niveau B1 validé par l'examen {\em Cambridge English Preliminary}),\newline
  Espagnol niveau débutant
}
%% \cvlanguage{Anglais}{CEFR niveau B1}{validé par l'examen «\,Cambridge English Preliminary\,»}
%% \cvlanguage{Espagnol}{niveau débutant}{}


\cvitem{Bureautique}{
  \LaTeX,
  LibreOffice
}

\cvitem{Autres}{
  Comptabilité,
  Droit,
  Économie,
  Gestion d'entreprises,
  Méthodes Agiles
}

\section{Formation}
\cventry{2016\,--\,actuel}{Préparation d'un diplôme Universitaire de Technologie en Informatique}{IUT}{\newline Aix-en-Provence}{}{2\textsuperscript{de} année en alternance}

\cventry{2016}{Baccalauréat Scientifique, série Sciences de l'Ingénieur}{}{}{\mbox{mention \em{bien}}}{Spécialité Informatique et Sciences du Numérique}

\section{Expérience professionnelle}
\cventry{2017}{Projet tutoré}
        {pédagogique, hexanôme}{}{}{
          Conception d'un service de calculs d'itinéraires basé sur les cartes OpenStreetMap.\newline
          Gestion de projet, et travail en groupe.\newline
          Réalisation d'une IHM avec la bibliothèque JavaFX.
        }

\cventry{2017\,-\,2018}{Chargé de projet}
        {Agence Boeki}{Aix-en-Provence}{alternance}{
          Réalisation d'un système de gestion de caisse en ligne avec le «\,framework\,» Symfony 3.
        }

\section{Centres d'intérêt}
\cvitem{\textbf{Événements}}{
  Hack to the Future,
  Nuit de l'Info 2017
}

\cvitem{\textbf{Veille technologique}}{
  GNU\,/\,Linux,
  programmation,
  libre\,/\,Open Source,
  Reddit,
  crypto-monnaies (Bitcoin),\newline
  communications sécurisées,
  vie privée
}

\cvitem{\textbf{Logiciel libre}}{
  Contributions au développement de logiciels libres
}

\cvitem{\textbf{Escalade}}{
  Pratique compétitive au niveau régional,
  diplôme d'Initiateur SAE délivré par la FFME
}

%% \clearpage

\section{Projets et réalisations}
%% \cventry{2016}{Projet web}
%%         {pédagogique, pentanôme}{}{}{
%%           Réalisation d'un site web en HTML5, CSS3 et PHP, utilisant Gettext pour traduire du contenu.
%%         }

\cventry{2016}{Projet C++}
        {pédagogique, quadrinôme}{}{}{
          Réalisation d'un jeu en terminal écrit en C++, utilisant la bibliothèque ncurses.
        }

\cventry{3 mois\\2016}{Projet d'ISN}
        {pédagogique continué en projet personnel, trinôme}{}{}{
          Réalisation d'un jeu écrit en C/C++, utilisant la bibliothèque graphique SDL.
          Nous avons poursuivi ce projet en dehors du cadre scolaire après l'avoir rendu.
        }


\end{document}
