\documentclass[11pt,a4paper]{moderncv}

\usepackage[french]{babel}
\usepackage[T1]{fontenc}
\usepackage[top=10mm, bottom=10mm, left=10mm, right=10mm]{geometry}
\usepackage[none]{hyphenat}
\usepackage[utf8]{inputenc}

\moderncvtheme[blue]{classic}

\firstname{Pierre}
\familyname{Gillet}
\address{125 allée des bastides}{83470, S\up{t}-Maximin-la-S\up{te}-Baume} 
\phone[mobile]{06\,09\,89\,44\,22}
\email{pierre.gillet@linuxw.info}
%% \homepage{floss.win}
\social[linkedin]{pierre-gillet-884393144}
\social[github]{pierregillet}
%% \social[stackexchange]{8394677}
%% \extrainfo{19\,ans}

%% \photo[64pt]{img/photo_profil.jpg}

%% \title{Je souhaite poursuivre mes études jusqu'au doctorat.}

\begin{document}

\maketitle

\vspace{-6mm}
\section{Projet professionnel}
Passionné d'Informatique et de culture libre,
je souhaite exercer le métier d'enseignant-chercheur en Informatique.

\section{Compétences et connaissances}
\textbf{Spécialisées}\newline
\cvitem{Langages}{
  \textbf{Python},
  \textbf{Java},
  \textbf{C++},
  PHP,
  Bash,
  JavaScript,
  C,
  Assembleur (MIPS),
  algorithmique
}

\cvitem{Frameworks}{
  \textbf{GTK+},
  \textbf{Qt},
  Bootstrap,
  Django,
  Flask,
  Symfony 3
}

\cvitem{Bibliothèques}{
  \textbf{SDL},
  \textbf{SFML},
  Boost,
  Bokeh,
  NumPy,
  PyGame
}

\cvitem{Développement}{
  \textbf{Design patterns},
  \textbf{Git},
  débuggers,
  UML,
  modèles conceptuels de données, 
  intégration continue,
  tests unitaires et fonctionnels,
  \textbf{GitHub}\,/\,GitLab\,/\,Gogs,
  Meson,
  GNU Make\,/\,Autotools
  %% Maven
}

\cvitem{Systèmes}{
  GNU\,/\,Linux (\textbf{Arch Linux}, Debian),
  Docker
  %% VirtualBox
}

\cvitem{Réseaux}{
  \textbf{Nginx},
  Apache 2,
  DNS,
  DHCP,
  routage
  %% Spanning Tree Protocol
}

\cvitem{Bases de données}{
  SQL,
  PL\,/\,SQL,
  MariaDB\,/\,MySQL\,/\,SQLite,
  Triggers,
  PHPMyAdmin
  %% MySQL Workbench
}

\cvitem{IDE\,/\,Éditeurs}{
  \textbf{IntelliJ IDEA},
  \textbf{PyCharm},
  \textbf{CLion},
  PhpStorm,
  Qt Creator,
  \textbf{Emacs},
  Vim
  %% Subliminal,
  %% Atom
}

\textbf{Associées\newline}
\cvitem{Langues}{
  Anglais (CEFR niveau B1 ({\em Cambridge English Preliminary}), Bachelor en Écosse),\newline
  Espagnol niveau débutant
}

\cvitem{Autres}{
  \textbf{\LaTeX},
  comptabilité,
  droit,
  économie,
  gestion d'entreprises,
  méthodes agiles
}

\section{Expérience professionnelle et projets personnels}
\cventry{2017\,-\,2018}{Chargé de projet}{Agence Boeki}{Aix-en-Provence}{alternance}{
  Réalisation d'un système de gestion de caisse en ligne avec le «\,framework\,» Symfony 3.
}

\cventry{2017}{OpenItinerary (projet tutoré)}{pédagogique, hexanôme}{}{}{
  Conception d'un service de calculs d'itinéraires basé sur les cartes OpenStreetMap.
}

\cventry{2016}{Projet C++}{pédagogique, quadrinôme}{}{}{
  Réalisation d'un jeu en terminal écrit en C++, utilisant la bibliothèque ncurses.
}

\cventry{2016}{Projet de terminale (ISN)}{pédagogique continué en projet personnel, binôme}{}{}{
  Réalisation d'un jeu écrit en C/C++, utilisant la bibliothèque graphique SDL.\newline
  Nous avons poursuivi le développement de ce projet en dehors du cadre scolaire après l'avoir rendu.
}

%% \cventry{2016}{Projet web}
%%         {pédagogique, pentanôme}{}{}{
%%           Réalisation d'un site web en HTML5, CSS3 et PHP, utilisant Gettext pour traduire du contenu.
%%         }

\section{Formation}
\cventry{2018--2019}{Bachelor en «\,Computer Games Technology\,»}{\newline University of the West of Scotland}{Paisley (Écosse)}{}{
  Année réalisée dans le cadre du programme d'échange européen Erasmus.
}

\cventry{2018}{DUT Informatique en alternance}{IUT AMU}{Aix-en-Provence}{}{}

\cventry{2016}{Baccalauréat Scientifique, série SI, spécialité ISN}{}{}{\mbox{mention \em{bien}}}{}

\section{Centres d'intérêt}
\cvitem{\textbf{Veille technologique}}{
  GNU\,/\,Linux,
  programmation,
  Libre et Open Source,
  Reddit,
  communications sécurisées,
  vie privée,
  science et industrie spaciale,
  transition énergétique
  %% crypto-monnaies (Bitcoin)
}

\cvitem{\textbf{Événements}}{
  Hack to the Future 2017,
  Nuit de l'Info 2017
}

\cvitem{\textbf{Logiciel libre}}{
  Contributions au développement de logiciels libres
}

\cvitem{\textbf{Escalade}}{
  Pratique compétitive au niveau régional,
  diplôme d'Initiateur SAE délivré par la FFME
}
        
\end{document}
