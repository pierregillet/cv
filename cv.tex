\documentclass[11pt,a4paper]{moderncv}
\usepackage[utf8]{inputenc}
\usepackage[francais]{babel}
\usepackage[T1]{fontenc}
\usepackage[top=0.6cm, bottom=0.6cm, left=1.0cm, right=1.0cm]{geometry}

%%%%%%%%%%%%%%%%%%%%%%%%%%%%%%%%%%%%%%%%%%%%%%%%%%%%%%%%%%%%

%% \makeatletter
%% \renewcommand*{\maketitle}{%
%%  \setlength{\maketitlewidth}{0.8\textwidth}
%%   \hfill
%%   \parbox{\maketitlewidth}{%
%%    \centering
%%     % name and title
%%     \namestyle{\@firstname~\@lastname}%
%%     \ifthenelse{\equal{\@title}{}}{}{\titlestyle{~|~\@title}}\\% \isundefined doesn't work on \@title, as LaTeX itself defines \@title (before it possibly gets redefined by \title) 
%%     % detailed information
%%     \addressfont\color{color2}%
%%     \ifthenelse{\isundefined{\@addressstreet}}{}{\addtomaketitle{\addresssymbol\@addressstreet}%
%%       \ifthenelse{\equal{\@addresscity}{}}{}{\addtomaketitle[~--~]{\@addresscity}}% if \addresstreet is defined, \addresscity and \addresscountry will always be defined but could be empty
%%       \ifthenelse{\equal{\@addresscountry}{}}{}{\addtomaketitle[~--~]{\@addresscountry}}%
%%       \flushmaketitle\@firstmaketitleelementtrue\\}%
%%     \collectionloop{phones}{% the key holds the phone type (=symbol command prefix), the item holds the number
%%       \addtomaketitle{\csname\collectionloopkey phonesymbol\endcsname\collectionloopitem}}%
%%     \ifthenelse{\isundefined{\@email}}{}{\addtomaketitle{\emailsymbol\emaillink{\@email}}}%
%%     \ifthenelse{\isundefined{\@homepage}}{}{\addtomaketitle{\homepagesymbol\httplink{\@homepage}}}%
%%     \collectionloop{socials}{% the key holds the social type (=symbol command prefix), the item holds the link
%%       \addtomaketitle{\csname\collectionloopkey socialsymbol\endcsname\collectionloopitem}}%
%%     \ifthenelse{\isundefined{\@extrainfo}}{}{\addtomaketitle{\@extrainfo}}%
%%     \flushmaketitle}\\[2.5em]}% need to force a \par after this to avoid weird spacing bug at the first section if no blank line is left after \maketitle
%% \makeatother

%%%%%%%%%%%%%%%%%%%%%%%%%%%%%%%%%%%%%%%%%%%%%%%%%%%%%%%%%%%%

\moderncvtheme[blue]{classic}
\nopagenumbers{}

\firstname{Pierre}
\familyname{Gillet}
\address{Cité Universitaire Arc-de-Meyran}{13100 Aix-en-Provence} 
\phone[mobile]{+33\,6\,09\,89\,44\,22}
\email{pierre.gillet@linuxw.info}
%% \homepage{floss.win}
\social[github]{pierregillet}
\social[linkedin]{pierre-gillet-884393144}
\extrainfo{02/10/1998}

\photo[64pt][0.4pt]{../photos/photo_cv_petite_retouchee.png}

\title{\parbox{11cm}{\large Étudiant en DUT Informatique, je recherche une entreprise afin de réaliser une alternance d'1 an à partir du 5 septembre 2017, avec 31 semaines en entreprise et 21 en Université.}}

\begin{document}

\maketitle

\vspace{-8mm}
\section{Projet professionnel}
\begin{center}
  \parbox{18cm}{Je souhaite poursuivre mes études au delà du DUT, jusqu'au niveau Bac+5 au moins.\newline
    Je désire me diriger vers du développement logiciel.}
\end{center}

\section{Compétences et connaissances}
\textbf{Spécialisées\newline}
\cvitem{Programmation}{
  \textbf{Python},
  \textbf{Java},
  \textbf{C++},
  JavaScript (bases),
  C,
  Assembleur (MIPS),
  algorithmique
}

\cvitem{Développement}{
  \textbf{Git (gestion de version)},
  débuggers,
  UML,
  intégration continue
}

\cvitem{Systèmes}{
  GNU/Linux (\textbf{Arch Linux}, Debian),
  \textbf{Nginx},
  VNC
}

\cvitem{Réseaux}{
  DNS,
  routage,
  Spanning Tree Protocol,
  DHCP
}

\cvitem{Bases de données}{
  SQL,
  PL/SQL,
  MariaDB/MySQL
}

\cvitem{IDE\,/\,Éditeurs}{
  \textbf{IntelliJ IDEA},
  \textbf{CLion},
  \textbf{PyCharm},
  Qt Creator,
  \textbf{Emacs},
  Vim
}

\cvitem{Outils}{
  Maven,
  Make,
  VirtualBox,
  \textbf{JavaFX} (IHM),
}

\textbf{Associées\newline}
\cvitem{Langues}{
  Anglais (CEFR niveau B1 validé par l'examen {\em Cambridge English Preliminary}),\newline
  Espagnol niveau débutant
}

\cvitem{Bureautique}{
  LibreOffice,
  \LaTeX
}

\cvitem{Autres}{
  Comptabilité,
  Droit,
  Économie
}

\section{Projets et réalisations}
\cventry{2017}{Projet tutoré}
        {pédagogique, équipe de 6}{}{}{
          Conception d'un service de calculs d'itinéraires basé sur les cartes OpenStreetMap.\newline
          Gestion de projet, et travail en groupe.\newline
          Réalisation d'une IHM avec la bibliothèque JavaFX.
        }

\cventry{2016}{Projet web}
        {pédagogique}{}{}{
          Réalisation d'un site web en HTML5 et CSS3 présentant le projet tutoré, faisant usage de JavaScript.
        }

\cventry{2016}{Projet C++}
        {pédagogique, équipe de 4}{}{}{
          Réalisation d'un jeu en terminal écrit en C++, utilisant la bibliothèque ncurses.
        }

\cventry{3 mois\\2016}{Projet C/C++}
        {pédagogique continué en projet personnel, trinôme}{}{}{
          Réalisation d'un jeu écrit en C/C++, utilisant la suite GNU Autotools et la bibliothèque graphique SDL.
          Nous avons poursuivi ce projet durant plusieurs mois après l'avoir rendu.
        }

\section{Formation}
\cventry{2016\,--\,actuel}{Préparation d'un diplôme Universitaire de Technologie en Informatique}{IUT}{\newline Aix-en-Provence}{}{}

\cventry{2016}{Baccalauréat Scientifique, série Sciences de l'Ingénieur}{}{}{\mbox{mention \em{bien}}}{Spécialité Informatique et Sciences du Numérique}

\section{Centres d'intérêt}
\cvitem{\textbf{Événements}}{
  Hack to the Future
}

\cvitem{\textbf{Veille technologique}}{
  GNU/Linux,
  programmation,
  libre/Open Source,
  Reddit,
  crypto-monnaies (Bitcoin),\newline
  communications sécurisées,
  privacy
}

\cvitem{\textbf{Cartographie}}{
  OpenStreetMap
}

\cvitem{\textbf{Logiciel libre}}{
  Contributions au développement de logiciels libres
}

\cvitem{\textbf{Escalade}}{
  Pratique compétitive au niveau régional,
  diplôme d'Initiateur SAE délivré par la FFME
}

\end{document}
